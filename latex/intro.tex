% !TEX root = CartaPerali_Report.tex

\section{Introduction}
\label{sec:introduction}
\noindent \textbf{Problem definition and relevance.} Keyword spotting is the problem of transcribing keyword utterance. Several voice interfaces (i.e. Alexa, Siri, Cortana) rely on keyword spotting for the detection of wake-up words that determine the beginning of an interaction. These systems are continuously listening for an audio input from the embedded microphones and, once a likely trigger phrase is detected, the interaction begins. 
In this paper keyword spotting is addressed as a closed-set classification problem, as most of the existing solutions. CNN-based architectures are deployed playing on different data preprocessing parameters and \mbox{Neural Network's} hyperparameters tuning.\\

\noindent \textbf{Paper contribution.} In this paper a keyword spotting system is implemented investigating Convolutional Neural Network with MFCC or spectrogram preprocess starting from the light model developed by Sainath and Parada \cite{sainath2015convolutional} and moving towards the neural attention model proposed by \cite{de2018neural}. \\

\noindent \textbf{Dataset description.} \cite{Warden-2018} The systems is trained on the \mbox{"{\it speech\_commands V2}"} dataset. It contains 105,829 utterances of 35 distinct words coming from 2,618 different persons, each one having the duration of at most one second, 16000 Hz, stored in {\it .wav} format files. The contained audio allow to reflect the trigger phrase task that a keyword spotting system aims at achieving and cope with noisy environments, poor quality recording tools or people taking in a natural way. \\ 

\noindent \textbf{Paper structure.} In Section~I it is introduced the problem and what is done in the paper. In Section~II it is described the related work and state of the art. The processing pipeline is described in Section~III. In Section~IV the data preprocessing and features extraction work is presented. In Section~V the learning frameworks are described . Section~VI describes the training analysis. Obtained results are presented in Section~VII.  Concluding remarks are provided in Section~VIII.

